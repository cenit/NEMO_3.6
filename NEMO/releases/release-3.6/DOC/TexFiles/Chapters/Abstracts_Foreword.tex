\documentclass[NEMO_book]{subfiles}
\begin{document}

% ================================================================
% Abstract (English / French)
% ================================================================

\chapter*{Abstract / R\'{e}sum\'{e}}

\vspace{-40pt}

{\small
The ocean engine of NEMO (Nucleus for European Modelling of the Ocean) is a primitive 
equation model adapted to regional and global ocean circulation problems. It is intended to 
be a flexible tool for studying the ocean and its interactions with the others components of 
the earth climate system over a wide range of space and time scales. 
Prognostic variables are the three-dimensional velocity field, a non-linear sea surface height, 
the \textit{Conservative} Temperature and the \textit{Absolute} Salinity. 
In the horizontal direction, the model uses a curvilinear orthogonal grid and in the vertical direction, 
a full or partial step $z$-coordinate, or $s$-coordinate, or a mixture of the two. 
The distribution of variables is a three-dimensional Arakawa C-type grid. 
Various physical choices are available to describe ocean physics, including TKE, and GLS vertical physics. 
Within NEMO, the ocean is interfaced with a sea-ice model (LIM or CICE), passive tracer and 
biogeochemical models (TOP) and, via the OASIS coupler, with several atmospheric general circulation models. 
It also support two-way grid embedding via the AGRIF software.

% ================================================================
% \vspace{0.5cm}

%Le moteur oc\'{e}anique de NEMO (Nucleus for European Modelling of the Ocean) est un 
%mod\`{e}le aux \'{e}quations primitives de la circulation oc\'{e}anique r\'{e}gionale et globale. 
%Il se veut un outil flexible pour \'{e}tudier sur un vaste spectre spatiotemporel l'oc\'{e}an et ses 
%interactions avec les autres composantes du syst\`{e}me climatique terrestre. 
%Les variables pronostiques sont le champ tridimensionnel de vitesse, une hauteur de la mer 
%lin\'{e}aire, la Temp\'{e}rature Conservative et la Salinit\'{e} Absolue. 
%La distribution des variables se fait sur une grille C d'Arakawa tridimensionnelle utilisant une 
%coordonn\'{e}e verticale $z$ \`{a} niveaux entiers ou partiels, ou une coordonn\'{e}e s, ou encore 
%une combinaison des deux. Diff\'{e}rents choix sont propos\'{e}s pour d\'{e}crire la physique 
%oc\'{e}anique, incluant notamment des physiques verticales TKE et GLS. A travers l'infrastructure 
%NEMO, l'oc\'{e}an est interfac\'{e} avec des mod\`{e}les de glace de mer (LIM ou CICE), 
%de biog\'{e}ochimie marine et de traceurs passifs, et, via le coupleur OASIS, \`{a} plusieurs 
%mod\`{e}les de circulation g\'{e}n\'{e}rale atmosph\'{e}rique. 
%Il supporte \'{e}galement l'embo\^{i}tement interactif de maillages via le logiciel AGRIF.
} 

% ================================================================
% Disclaimer
% ================================================================
\chapter*{Disclaimer}

Like all components of NEMO, the ocean component is developed under the CECILL license, 
which is a French adaptation of the GNU GPL (General Public License). Anyone may use it 
freely for research purposes, and is encouraged to communicate back to the NEMO team 
its own developments and improvements. The model and the present document have been 
made available as a service to the community. We cannot certify that the code and its manual 
are free of errors. Bugs are inevitable and some have undoubtedly survived the testing phase. 
Users are encouraged to bring them to our attention. The author assumes no responsibility 
for problems, errors, or incorrect usage of NEMO.

 \vspace{1cm}
NEMO reference in papers and other publications is as follows:
 \vspace{0.5cm}

Madec, G., and the NEMO team, 2008: NEMO ocean engine. 
\textit{Note du P\^ole de mod\'{e}lisation}, Institut Pierre-Simon Laplace (IPSL), France, 
No 27, ISSN No 1288-1619.\\


 \vspace{0.5cm}
Additional information can be found on \href{http://www.nemo-ocean.eu/}{nemo-ocean.eu} website.
 \vspace{0.5cm}

\end{document}
